\documentclass{article}
\usepackage{geometry}
\geometry{a4paper, top=3cm, left=3cm, right=3cm, bottom=3cm}
\usepackage[utf8]{inputenc}
\usepackage{mathtools} 
\usepackage{float}
\usepackage{amssymb}
\usepackage{mathrsfs}
\usepackage{qtree}[]
\usepackage{titlesec}
%\titleformat{\subsection}[runin]{}{}{}{}[]
\usepackage{titlesec}
\titlelabel{\thetitle.\quad}
\usepackage{palatino}
%%% CRUCIAL COMMANDS FOR ME $$$$
\usepackage{natbib}
\renewcommand{\bibsection}{}

\title{\sc Impacts of Piped Water on Urban Development, Public Health, and Consumer Welfare: \\ Lessons from Manila Water}

\begin{document}

\maketitle

\large 

%\vspace{5mm}
%\begin{center}
%{ \bf ECON2410} \\
%{ \sc Stefano Polloni} \\
%{ \bf 2\textsuperscript{nd} Research Sketch} \\
%\end{center}

% \vspace{4mm}

\section*{Principal Investigator}

William Violette \\
PhD Candidate in Economics \\
Brown University \\
Providence, Rhode Island, USA\footnote{Contact information: william\_violette@brown.edu}


\section*{Motivation}

Through innovative policies to improve infrastructure, Manila Water has dramatically expanded water access and quality throughout its service area.  Despite these large improvements, few studies have quanitified exactly how piped water expansion has impacted public health, driven business growth, and improved consumer welfare.  This research aims to fill this gap by analyzing the impacts of both greater access as well as improved reliability and quality of piped water.

Beyond helping to motivate improved water policies in the Philippines, this study will have broad application for water systems across the developing world.  As urban areas continue to grow at rapid rates, policymakers face important tradeoffs in allocating resources between different urban development projects.  The experience of Manila Water provides an unique setting to investigate both the role of piped water access as well as improvements in quality and reliability for urban development.

\section*{Research Questions and Approach}

This research project will can be divided into two main themes.  The first theme will investigate how water access impacts health, business growth, and employment.  The second theme will focus more specifically on reliability and quality in order to estimate how these factors benefit consumer welfare as well as urban development.  Both themes together will provide a comprehensive picture of the impacts of improved water access on urban development.

\subsection{Impacts of Piped Water Access on Urban Development}

\subsection*{Methods}

To accurately quantify the impacts of piped water access, I will compare public health measures, business growth, and employment growth in small geographic areas before and after receiving piped water access.  I will also go one step further and compare these changes in outcomes to nearby areas that have not yet received water access.  This method will allow me to isolate the true causal impact of water access on urban development indicators.

\subsection*{Data}

I am working to assemble a wide range of datasets in order to perform this analysis with precision in both geographic and time dimensions.  I list the data sources below:

\begin{itemize}
\item I hope to work with Manila Water to construct detailed maps of piped water rollout throughout the service area since the concession agreement.  I will use this information to identify exactly when each specific area first received piped water access.  While much of the analysis will take place at the Barangay level, I hope to gather data at a finer geographic resolution in order to estimate some outcomes as precisely as possible.
\item Health outcomes such as number of severe diarrhea cases will come from the Field Health Services Information System.  This dataset is collected by the Department of Health and includes the most comprehensive set of health outcomes for the Philippines.
\item Business growth data will be computed from business registration information, including business addresses, type of business, and registration date.  These data are available from the Philippines Department of Trade and Industry.  I will also compute employment data from a combination of Census data as well as National Labor Force Surveys.  Both of these datasets are available from the National Statistics Office.
\end{itemize}

\subsection{Impacts of Improved Reliability and Quality on Consumer Welfare and Urban Development}

\subsection*{Methods}

Estimating the value to consumers of water reliability and quality will require more sophisticated statistical methods as well as more detailed data on individual water consumption.  However, by providing a clear estimate of the value of infrastructure improvements for consumers, this study will provide useful information to help Manila water as well as policymakers decide how much to invest in improved infrastructure.  This aspect of the analysis will take place in two parts.

First, in order to estimate the value of reliability and quality to consumers, I will compute how different pipe replacement and non-revenue reduction policies improve reliability and quality in specific parts of the system.  I will then statistically estimate how individual consumption levels change for consumers that directly benefit from these improvements as compared to consumers that are unaffected.  In order to place a dollar value on these consumer benefits, I will estimate the value of water consumption by investigating how sensitive water consumption is to price for different consumers.  I can combine this price sensitivity estimate with earlier estimates to recover the total value of reliability and quality to consumers.

Second, I will repeat the previous water access analysis except that I will replace water access indicators with water reliability and water quality indicators.  I will compare changes in health, employment, and business indicators for areas that benefitted from pipe replacement projects to areas that did not benefit.  This method will recover the average causal impact of quality and reliability improvements on important urban development indicators.

\subsection*{Data}

\begin{itemize}
\item I hope to work with your team to assemble a set of anonymized records of individual connections and consumption data that stretch before and after different infrastructure projects.  The ideal data would include monthly values for the amount of water consumption, the prices paid, and the location of each individual connection.  Individual data is especially necessary for this analysis in order to ensure that the estimates accurately reflect consumers' true valuation of water consumption as opposed to other trends that may affect aggregate consumption patterns.  I fully recognize that when working with individual level data, it is essential to fully protect the privacy and security of consumers, so I would work closely with your colleagues to anonymize the identities of consumers, protect the data on secure hard drives, and only report aggregated statistics in any publications.
\item I also hope to assemble a dataset that documents exactly where and when infrastructure upgrades were implemented by Manila Water as well as the extent to which these upgrades improved reliability and water quality.  These measures will provide crucial variation in water quality levels as well as reliability.
\item To estimate broader impacts of reliability and quality, I will combine data on pipe infrastructure projects with the previously described measures of business growth, employment growth, and public health.
\end{itemize}

\section*{Partnership with Manila Water and Project Timeline}

I have received research funding for this project from the Population Studies and Traning Center at Brown University as well as from the Integrative Graduate Education and Research Traineeship through the National Science Foundation.  This funding will support an extended trip to Manila starting in June to gather datasets from various government agencies as well as to work closely with Manila Water to clean, organize, and analyze data for this study.  While working with your colleagues at Manila Water, I would be excited to use my quantitative training to answer important questions about that would directly help operations at Manila Water.  Possible topics include water demand analysis and computing the expected return of different water infrastructure projects.

%\subsection*{References}
%\bibliographystyle{abbrvnat}
%\setcitestyle{authoryear,open={((},close={))}}
%\bibliography{bib}


\end{document}

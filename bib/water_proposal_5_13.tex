\documentclass{article}
\usepackage{geometry}
\geometry{a4paper, top=3cm, left=3cm, right=3cm, bottom=3cm}
\usepackage[utf8]{inputenc}
\usepackage{mathtools} 
\usepackage{float}
\usepackage{amssymb}
\usepackage{mathrsfs}
\usepackage{qtree}[]
\usepackage{titlesec}
%\titleformat{\subsection}[runin]{}{}{}{}[]
\usepackage{titlesec}
\titlelabel{\thetitle.\quad}
\usepackage{palatino}
%%% CRUCIAL COMMANDS FOR ME $$$$
\usepackage{natbib}
\renewcommand{\bibsection}{}

\title{\sc Impacts of Piped Water on Urban Development, Public Health, and Consumer Welfare: \\ Lessons from Manila Water}

\begin{document}

\maketitle

\large 

%\vspace{5mm}
%\begin{center}
%{ \bf ECON2410} \\
%{ \sc Stefano Polloni} \\
%{ \bf 2\textsuperscript{nd} Research Sketch} \\
%\end{center}

% \vspace{4mm}

\section*{Principal Investigator}

William Violette \\
PhD Candidate in Economics \\
Brown University \\
Providence, Rhode Island, USA\footnote{Contact information: william\_violette@brown.edu}

\section*{Motivation}

Through innovative policies to improve infrastructure, Manila Water has dramatically expanded water access from 50% coverage before the concession agreement to 90% coverage in 2012.  Manila Water has also reduced non-revenue water from 63% to 11% over the same time interval, leading to significant gains in water quality and reliability. Despite these large improvements, few studies have quantified exactly how piped water expansion has impacted public health, driven business growth, and improved consumer welfare, which can be measured by greater willingness-to-pay.  This research aims to fill this gap by analyzing the impacts of expanded access, as well as improved reliability and quality of piped water.

Beyond helping to inform efforts to improve water policies in the Philippines, this study will have broad application for decision-makers across the developing world.  As urban areas continue to grow at rapid rates, policymakers face important tradeoffs in allocating resources between different urban development priorities.  The experience of Manila Water provides a unique setting to investigate both the role of piped water access as well as improvements in quality and reliability for urban development.

\section*{Research Questions and Approach}

This research project has two main objectives: (1) demonstrate how improved or expanded access to safe water impacts health, business growth, and employment; and (2)  evaluate how improved reliability and water quality benefit consumer willingness-to-pay and urban development.  Taken together, both objectives will provide an understanding of the relationship between improved access and urban development.

\subsection{Impacts of Piped Water Access on Urban Development}

\subsection*{Methods}

To accurately quantify the impacts of access to piped water, the investigation will compare public health measures, business growth, and employment growth in small geographic areas before and after receiving piped water.  The study will also compare the changes in these outcomes to those in nearby areas that have not yet received access to piped water.  This method will isolate the true causal impact of water access on standard indicators of urban development.

\subsection*{Data}

To support analysis, the investigation will assemble a wide range of datasets in both geographic and temporal dimensions drawn from public sources and information from Manila Water.   Data sources are as follows:

\begin{itemize}
\item Manila Water will provide data across its service area, which will result in detailed maps of piped water rollout since the concession agreement.  This information will help to identify the timing for when each specific service area first received piped water access.  While much of the analysis will take place at the Barangay level, the plan is to gather data at a finer geographic resolution to estimate some outcomes as precisely as possible.
\item Health outcomes such as number of severe diarrhea cases will come from the Field Health Services Information System.  The Department of Health collects this dataset, which includes the most comprehensive set of health outcomes for the Philippines.
\item Business growth data will be computed from business registration information, including business addresses, type of business, and registration date.  These data are available from the Philippines Department of Trade and Industry.  The study will also compute employment data from a combination of Census data as well as National Labor Force Surveys.  Both of these datasets are available from the National Statistics Office.
\end{itemize}

\subsection{Impacts of Improved Reliability and Quality on Consumer Welfare and Urban Development}

\subsection*{Methods}

Estimating the value to consumers of water reliability and quality will require more sophisticated statistical methods as well as more detailed data on individual water consumption.  By providing a clear estimate of the value of infrastructure improvements for consumers, however, this study will provide useful information on willingness-to-pay to help Manila Water, as well as policymakers, decide how much to invest in improved infrastructure planning and investment.  This aspect of the analysis will take place in two parts.

First, to estimate the value of reliability and quality to consumers, the study will compute how different pipe replacement and non-revenue reduction policies improve reliability and quality in specific parts of the system.  The next step will be statistically estimating how individual consumption levels change for consumers that directly benefit from these improvements as compared to consumers that are unaffected.  The relationship between these two estimates will recover the sensitivity of water consumption to water quality and reliability.  In order to place a dollar value on these consumer benefits, it will then be necessary to estimate the value of water consumption by investigating how sensitive water consumption is to price for different consumers.  This procedure will result in a clear estimate of the willingness-to-pay for improvements in water infrastructure.

Second, the study will repeat the previous analysis of water access, instead using water reliability and water quality indicators as variables of interest.  The comparison of changes in health, employment, and business indicators for areas that benefitted from pipe replacement projects to areas that had not yet benefitted will produce an estimate of the average causal impact of quality and reliability improvements on important urban development indicators.

\subsection*{Data}

\begin{itemize}
\item  Water consumption data necessary for this part of the study would include monthly values for the amount of water consumption, the prices paid, and the location of each individual connection.  Individual data is especially necessary for this analysis in order to ensure that the estimates accurately reflect consumers' true valuation of water consumption as opposed to other trends that may affect aggregate consumption patterns.   Manila Water will work to assemble a set of anonymized records of individual connections and consumption data that stretches before and after different infrastructure projects.  When working with individual level data, it is essential to fully protect the privacy and security of consumers.  Therefore, the study will work closely with colleagues at Manila Water to anonymize the identities of consumers, protect the data on secure hard drives, and only report aggregated statistics in any publications.
\item Manila Water will also provide data that documents exactly where and when infrastructure upgrades were implemented as well as the extent to which these upgrades improved reliability and water quality.  These measures will provide crucial variation in water quality levels as well as reliability, which the study can then link to information on water consumption.
\item Previously described measures of business growth employment growth, and public health will also be essential to this segment of the analysis in providing important urban development indicators to link to quality and reliability improvements.
\end{itemize}

\section*{Partnership with Manila Water and Project Timeline}

The study has received research funding from the Population Studies and Training Center at Brown University as well as from the Integrative Graduate Education and Research Traineeship through the National Science Foundation.  This funding will support an extended trip by William Violette to Manila starting in June to gather data from various government agencies as well as to work closely with Manila Water to organize and analyze data, and identify areas for optimizing operations, as appropriate.   Possible topics include water demand analysis and computing the expected return of different water infrastructure projects.



%\subsection*{References}
%\bibliographystyle{abbrvnat}
%\setcitestyle{authoryear,open={((},close={))}}
%\bibliography{bib}


\end{document}
